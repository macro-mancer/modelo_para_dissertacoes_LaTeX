% ----------------------------------------------------------
\chapter[Introdução]{Introdução}

Parte inicial do texto, que tem como objetivo elucidar os principais conceitos necessários para compreender os objetivos da pesquisa e outros elementos necessários para apresentar o tema do trabalho \citep{the_craft_of_research}. É importante que a introdução seja clara e convincente. 

A equipe de desenvolvimento e manutenção da classe PPGCCUFSJ e do modelo para teses e dissertações em \LaTeX\ utilizando a classe PPGCCUFSJ é integralmente composta pelas pessoas listadas abaixo. Atualmente, o propósito da equipe é garantir a sustentabilidade deste modelo, tendo autonomia para implementar novos recursos, efetuar compatibilizações necessárias em decorrência de alterações de normas da ABNT e/ou normas e padrões estabelecidos pela comissão de pós-graduação da UFSJ. 

\textbf{Programação}

\begin{itemize}
	\item Vinicius H. S. Durelli -- \url{durelli@ufsj.edu.br}
\end{itemize}

\textbf{Normalização e Padronização}

\begin{itemize}
	\item Vinicius H. S. Durelli -- \url{durelli@ufsj.edu.br}
\end{itemize}
	
O objetivo do presente trabalho é apresentar a classe PPGCCUFSJ e o modelo para teses e dissertações em \LaTeX\ utilizando a classe PPGCCUFSJ. 
A expectativa é que a classe PPGCCUFSJ e o modelo proposto auxiliem no aprimoramento da qualidade dos trabalhos acadêmicos produzidos pelos alunos de pós-graduação da UFSJ.
	
