%!TEX root = /Users/Viny/Desktop/Thesis/thesis_main.tex

% CONTEXT/BACKGROUND: 
\lettrine{H}{igh-level} language virtual machines (HLL~VMs) have been playing a 
key role as a mechanism for implementing portable programming languages. 
Languages that run on these execution environments have many advantages over languages that are compiled to native code. 
These advantages have led HLL~VMs to gain broad acceptance in both academy and industry. 
% OBJECTIVE: 
However, 
much of the research in this area has been devoted to boosting the 
performance of these execution environments. 
Few efforts have attempted to introduce features that automate or 
facilitate some software engineering activities, 
including software testing. 
This research argues that 
HLL~VMs provide a reasonable basis for building an integrated software testing environment. 
% METHOD: 
To this end, 
two software testing features that build on the characteristics of  
a Java virtual machine (JVM) were devised. 
The purpose of the first feature is to automate weak mutation. 
Augmented with mutation support, 
the chosen JVM achieved speedups of as much as 89\% in comparison to a strong mutation tool. 
To support the testing of concurrent programs, 
the second feature is concerned with enabling the deterministic re-execution of Java programs and 
exploration of new scheduling sequences.
% RESULTS:  
% CONCLUSION:

\smallskip
\noindent \textbf{Keywords ---} software testing; mutation testing; weak mutation; record-and-playback mechanism; Maxine~VM; Java virtual machine.

% TO USE:
% The _recent_surge_in_ virtualization-related research has made its way not only into the security field but also into the malware scene, resulting in the creation of a new class of rootkits.

% EXAMPLE: "This dissertion investigates the possibility to improve the quality of text composi- tion. Two typographic extensions were examined: margin kerning and composing with font expansion."
